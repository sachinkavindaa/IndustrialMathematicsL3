\chapter*{Appendix}


\textbf{\Large Code 01}\\

\begin{framed}
\begin{verbatim}

package main.java.com.iim;
import java.io.IOException;
import java.io.PrintWriter;
import com.opencsv.CSVWriter;
import com.opencsv.CSVReader;
import java.io.File;
import java.io.FileNotFoundException;
import java.io.FileWriter;
import java.util.List;
import java.util.ArrayList;
import java.lang.Math;
import java.text.Format;
import com.iim.ode;


public final class Euler {
 

// Global variable for values obatain from w,x,y and z. 
// Global variable can be accesed anywhere within the class. 

private double _w;
private double _x;
private double _y;
private double _z;

    public Euler(double w, double x, double y, double z){


        _w = w;
        _x = x;
        _y = y;
        _z = z;
    }


// Defining the ODE equation of STAT1 concentration 
    public static double dw(double w, double x, double y, double z) {

        double lamdaS1 = 1;
        double S = 1;
        double k_1 = 4;
        double k_2 = 1;
        double alpha = 1.5;


        return  (lamdaS1*S) +( (k_1*Math.pow((k_2),2))/ (Math.pow((k_2),2) 
        + (alpha*Math.pow((x),2))) ) - w;
    }

// Defining the ODE equation of STAT3 concentration     
    public static double dx(double w, double x, double y, double z) {

        double lamda_k = 1;
        double lamda_J = 4;
        double J = 1;
        double K = 5;
        double lamdaS2 = 1;
        double k_3 = 4;
        double k_4 = 1;
        double beta = 1;
        double S = 1;

        return ((lamda_k + (lamda_J* J)) / (K+ (lamdaS2* S))) 
        + (((k_3)*Math.pow((k_4),2)) / ((Math.pow((k_4),2))
        + beta*Math.pow((w),2)))-x;
    }


// Defining the ODE equation of Bcl-2 concentration  
    public static double dy(double w, double x, double y, double z) {
        double lamda_1 = 0.2;
        double k_5 = 1;
        double k_6 = 1;
        double gemma = 1;
        double lamda_3 = 1.2;
        double mu_B = 1.2;

        return lamda_1+ (k_5*Math.pow((k_6),2)) / (Math.pow((k_6),2)+ 
        (gemma *Math.pow((w),2))) + 
        (lamda_3* x) - (mu_B*y);
    }

// Defining the ODE equation of BAX concentration     
    public static double dz(double w, double x, double y, double z) {
        double lamda_2 = 0.2;
        double k_7 = 4;
        double k_8 = 1;
        double delta = 1;
        double mu_X = 5;
        

        return lamda_2 + (k_7*Math.pow((k_8),2)) / (Math.pow((k_8),2) + 
        (delta*Math.pow((y),2)))- (mu_X*z);
    }



    public double getw(){
        return _w;
    }

    public double getx(){
        return _x;
    }
    
    public double gety(){
        return _y;
    }
    
    public double getz(){
        return _z;
    }
    


    public static void main(String[] args) throws FileNotFoundException {


        // Deifining the initial points 

            double w = 2.43;   //STAT1 level
            double x = 1.38;   //STAT3 Level
            double y = 10;     //Bcl-2 level
            double z = 351;    //BAX level
            double dt = 0.1;   //Step size
            double time = 10;  //Time duration
            double iteration = time/dt;   //Iteration


        // All data write into CSV file    
    
            File CSVfile = new File("Euler11.csv");
    
            ArrayList<Euler> values= new  ArrayList<>();

        // Values is the arraylist created to store multiple values 
        // That are stored using the value object in the line 138
        // Value object has 4 values namely w,x,y and z

            PrintWriter pw = new PrintWriter(CSVfile);

        // Use Euler's method to numerically solve ODE

            for (int i = 0; i < iteration; i++) {
    
                double dw = dw(w, x, y, z);
                double dx = dx(w, x, y, z);
                double dy = dy(w, x, y, z); 
                double dz = dz(w, x, y, z);
                w = w + dw*dt;
                x = x + dx*dt;
                y = y + dy*dt;
                z = z + dz*dt;
            
            
    
                Euler value= new Euler(w, x, y, z);
                values.add(value);
    
                if(i%1000 ==0){
                    System.out.println("w: "+w+" x: "+x+
                    " y: "+y+" z: "+z);
                }

            }

            String col1 = "w";
            String col2 = "x";
            String col3 = "y";
            String col4 = "z";
            
            pw.printf("%s,%s,%s,%s",col1,col2,col3,col4 ); 
            pw.print("\n");
            
            for(Euler reading: values){

            pw.printf("%f,%f,%f,%f",reading.getw(),reading.getx()
            ,reading.gety(),reading.getz());

        // The 4 values obtained for w,x,y and z are printed in the
        
        CSV file.

            pw.print("\n");


            
        // The printwriter has to be closed in 
        oder for the code to finish execution of the CSV file creation.
        }
        pw.close();
    
        }
    }
\end{verbatim}
\end{framed}

\textbf{\Large Code 02}\\

\begin{framed}
\begin{verbatim}
package main.java.com.iim;
import java.io.IOException;
import java.io.PrintWriter;
import com.opencsv.CSVWriter;
import com.opencsv.CSVReader;
import java.io.File;
import java.io.FileNotFoundException;
import java.io.FileWriter;
import java.util.List;
import java.util.ArrayList;
import java.lang.Math;
import java.text.Format;
import com.iim.ode;


public final class runge {
 

// Global variable for values obatain from w,x,y and z. 
Global variable can be accesed anywhere within the class. 

private double _w;
private double _x;
private double _y;
private double _z;

    public runge(double w, double x, double y, double z){
    
        _w = w;
        _x = x;
        _y = y;
        _z = z;
    }
    public static double dw(double w, double x, double y, double z) {

        double lamdaS1 = 1;
        double S = 1;
        double k_1 = 4;
        double k_2 = 1;
        double alpha = 1.5;


        return  (lamdaS1*S) +( (k_1*Math.pow((k_2),2))/
        
        (Math.pow((k_2),2) + (alpha*Math.pow((x),2))) ) - w;
    }

    public static double dx(double w, double x, double y, double z) {

        double lamda_k = 1;
        double lamda_J = 4;
        double J = 1;
        double K = 5;
        double lamdaS2 = 1;
        double k_3 = 4;
        double k_4 = 1;
        double beta = 1;
        double S = 1;

        return ((lamda_k + (lamda_J* J)) / 
        (K+ (lamdaS2* S))) + (((k_3)*Math.pow((k_4),2)) / 
        ((Math.pow((k_4),2))+ (beta*Math.pow((w),2))))-x;
    }

    public static double dy(double w, double x, double y, double z) {
        double lamda_1 = 0.2;
        double k_5 = 1;
        double k_6 = 1;
        double gemma = 1;
        double lamda_3 = 1.2;
        double mu_B = 1.2;

        return lamda_1+ ((k_5*Math.pow((k_6),2)) / (Math.pow((k_6),2)+
        (gemma *Math.pow((w),2)))) +  (lamda_3* x) - (mu_B*y);
    }

    public static double dz(double w, double x, double y, double z) {
     
        double lamda_2 = 0.2;
        double k_7 = 4;
        double k_8 = 1;
        double delta = 1;
        double mu_X = 5;
        
        return lamda_2 + ((k_7*Math.pow((k_8),2)) / (Math.pow((k_8),2) + 
        (delta*Math.pow((y),2))))- (mu_X*z);
    }

    public double getw(){
        return _w;
    }

    public double getx(){
        return _x;
    }
    
    public double gety(){
        return _y;
    }
    
    public double getz(){
        return _z;
    }
    
    public static void main(String[] args) throws FileNotFoundException {

        //initial points 
        double w = 2.43;
        double x = 1.38;
        double y = 10;
        double z = 351;
        double dt = 0.1; 
        double time = 10;
        double iteration = time/dt;

        // ode ODE = new ode();
        
        File CSVfile = new File("readingsrunge11.csv");

        ArrayList<runge> values= new  ArrayList<>();
       
        for (int i = 0; i < iteration; i++) {

            double k1w = dw(w, x, y, z) * dt;
            double k1x = dx(w, x, y, z) * dt;
            double k1y = dy(w, x, y, z) * dt;
            double k1z = dz(w, x, y, z) * dt;
    
            double k2w = dw(w + k1w/2, x + k1x/2, y + k1y/2, z + k1z/2) * dt;
            double k2x = dx(w + k1w/2, x + k1x/2, y + k1y/2, z + k1z/2) * dt;
            double k2y = dy(w + k1w/2, x + k1x/2, y + k1y/2, z + k1z/2) * dt;
            double k2z = dz(w + k1w/2, x + k1x/2, y + k1y/2, z + k1z/2) * dt;
    
            double k3w = dw(w + k2w/2, x + k2x/2, y + k2y/2, z + k2z/2) * dt;
            double k3x = dx(w + k2w/2, x + k2x/2, y + k2y/2, z + k2z/2) * dt;
            double k3y = dy(w + k2w/2, x + k2x/2, y + k2y/2, z + k2z/2) * dt;
            double k3z = dz(w + k2w/2, x + k2x/2, y + k2y/2, z + k2z/2) * dt;
    
            double k4w = dw(w + k3w, x + k3x, y + k3y, z + k3z) * dt;
            double k4x = dx(w + k3w, x + k3x, y + k3y, z + k3z) * dt;
            double k4y = dy(w + k3w, x + k3x, y + k3y, z + k3z) * dt;
            double k4z = dy(w + k3w, x + k3x, y + k3y, z + k3z) * dt;

        // update variables
        w =w+  (k1w + 2*k2w + 2*k3w + k4w) / 6;
        x =x+  (k1x + 2*k2x + 2*k3x + k4x) / 6;
        y =y+  (k1y + 2*k2y + 2*k3y + k4y) / 6;
        z =z+  (k1z + 2*k2z + 2*k3z + k4z) / 6;
    
        // store values in cancer object
        runge value= new runge(w, x, y, z);
        values.add(value);
            
        }
            String col1 = "w";
            String col2 = "x";
            String col3 = "y";
            String col4 = "z";
            
            pw.printf("%s,%s,%s,%s",col1,col2,col3,col4 ); 
            pw.print("\n");
            
            for(runge reading: values){
            pw.printf("%f,%f,%f,%f",reading.getw(),reading.getx(),
            reading.gety(),reading.getz());
  
            pw.print("\n");
        }
        pw.close();  
    }
}
\end{verbatim}
\end{framed}



\newpage

\textbf{\Large Code 03}

\begin{framed}
\begin{verbatim}

#Import libraries

import pandas as pd
import numpy as np
import matplotlib.pyplot as plt

#Uploading the CSV files 

df1 = pd.read_csv("/content/Euler11.csv", error_bad_lines=False)
df2 = pd.read_csv("/content/readingsrunge11.csv") 

#Creating the Time column

value = 0.1
df1["Time"] = df1.index * value
df2["Time"] = df2.index * value

#Plotting the Euler's results

fig = plt.figure()
ax = fig.add_subplot(111)
ax.plot(df1["Time"],df1["w"], 'r-', linestyle='solid', linewidth=2.5, 
label='STAT1')
ax.plot(df1["Time"],df1["x"],'b-', linestyle='dashed', linewidth=2.5, 
label='STAT3')
ax.plot(df1["Time"],df1["y"],'g-', linestyle='dotted', linewidth=2.5, 
label='BCL-2')
ax.plot(df1["Time"],df1["z"],'y-', linestyle='dashdot', linewidth=2.5, 
label='BAX')
ax.legend()
ax.set_title("Behaviour in time using Euler method")
ax.set_xlabel("Time")
ax.set_ylabel("Concentration")
ax.grid()
ax.legend(loc='best')
plt.show()

#Plotting the Runge kutta's results

fig = plt.figure()
ax = fig.add_subplot(111)
ax.plot(df2["Time"],df2["w"], 'r-', linestyle='solid', linewidth=2.5, 
label='STAT1')
ax.plot(df2["Time"],df2["x"],'b-', linestyle='dashed', linewidth=2.5, 
label='STAT3')
ax.plot(df2["Time"],df2["y"],'g-', linestyle='dotted', linewidth=2.5, 
label='BCL-2')
ax.plot(df2["Time"],df2["z"],'y-', linestyle='dashdot', linewidth=2.5, 
label='BAX')
ax.legend()
ax.set_title("Behaviour in time using Runge Kutta fourth order method")
ax.set_xlabel("Time")
ax.set_ylabel("Concentration")
ax.grid()
ax.legend(loc='best')
plt.show()


#Plotting the STAT1 results

fig = plt.figure()
ax = fig.add_subplot(111)
ax.plot(df1["Time"],df1["w"], 'r-', linestyle='solid',linewidth=2.5, 
label='Euler')
ax.plot(df2["Time"],df2["w"],'b-', linestyle='dashed',linewidth=2.5, 
label='Runge Kutta')
ax.legend()
ax.set_title("Behaviour of STAT1 in time")
ax.set_xlabel("Time")
ax.set_ylabel("STAT1")
ax.grid()
ax.legend(loc='best')
plt.show()

#Plotting the STAT3 results

fig = plt.figure()
ax = fig.add_subplot(111)
ax.plot(df1["Time"],df1["x"], 'r-', linestyle='solid',linewidth=2.5, 
label='Euler')
ax.plot(df2["Time"],df2["x"],'b-', linestyle='dashed',linewidth=2.5, 
label='Runge Kutta')
ax.legend()
ax.set_title("Behaviour of STAT3 in time")
ax.set_xlabel("Time")
ax.set_ylabel("STAT3")
ax.grid()
ax.legend(loc='best')
plt.show()


#Plotting the Bcl-2 results


fig = plt.figure()
ax = fig.add_subplot(111)
ax.plot(df1["Time"],df1["y"], 'r-', linestyle='solid',linewidth=2.5, 
label='Euler')
ax.plot(df2["Time"],df2["y"],'b-', linestyle='dashed',linewidth=2.5, 
label='Runge Kutta')
ax.legend()
ax.set_title("Behaviour of BCL-2 in time")
ax.set_xlabel("Time")
ax.set_ylabel("BCL-2")
ax.grid()
ax.legend(loc='best')
plt.show()

#Plotting the BAX results

fig = plt.figure()
ax = fig.add_subplot(111)
ax.plot(df1["Time"],df1["z"], 'r-', linestyle='solid',linewidth=2.5, 
label='Euler')
ax.plot(df2["Time"],df2["z"],'b-', linestyle='dashed',linewidth=2.5, 
label='Runge Kutta')
ax.legend()
ax.set_title("Behaviour of BAX in time")
ax.set_xlabel("Time")
ax.set_ylabel("BAX")
ax.grid()
ax.legend(loc='best')
plt.show()


\end{verbatim}
\end{framed}

\newpage

\textbf{\Large Code 04}

\begin{framed}
\begin{verbatim}

import numpy as np
from scipy.optimize import root

def H1(s, S1, S3, B, X):
    return (lambda_S1 * S) + (k1 * k2**2 / (k2**2 + alpha * S3**2)) - S1

def H2(s, S1, S3, B, X):
    return ((lambda_k + lambda_J * J) / (K + lambda_S2 * S)) + 
    (k3 * k4**2 / (k4**2 + beta * S1**2)) - S3

def H3(s, S1, S3, B, X):
    return lambda_1 + (k5 * k6**2 / (k6**2 + gamma * S1**2)) - 
    
    (lambda_3 * S3) - (mu_B * B)

def H4(s, S1, S3, B, X):
    return lambda_2 + (k7 * k8**2 / (k8**2 * delta * B**2)) - (mu_X * X)

def equations(s):
    S1, S3, B, X = s
    return H1(s, S1, S3, B, X), H2(s, S1, S3, B, X), 
    H3(s, S1, S3, B, X), H4(s, S1, S3, B, X)

# Set the parameters
lambda_S1 = 1
S = 0
k1 = 4
k2 = 1
alpha = 1.5
lambda_k = 1
lambda_J = 4
J = 1
K = 5
lambda_S2 = 1
k3 = 4
k4 = 1
beta = 1
S = 1
lambda_1 = 0.2
k5 = 1
k6 = 1
gamma = 1
lambda_3 = 1.2
mu_B = 1.2
lambda_2 = 0.2
k7 = 4
k8 = 1
delta = 1
mu_X = 5

# Initial guess for the state variables
s0 = [2.43 , 1.38 , 10, 351]

# Finding the equilibrium points using root function
equil_points = root(equations, s0)

S1_eq, S3_eq, B_eq, X_eq = equil_points.x

print("Equilibrium points: 
S1 = {0}, S3 = {1}, B = {2}, X = {3}".format(S1_eq, S3_eq, B_eq, X_eq))

\end{verbatim}
\end{framed}



\\




















 

































































































