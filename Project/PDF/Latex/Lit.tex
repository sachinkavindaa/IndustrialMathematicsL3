\chapter{Literature Review}
\label{chap:01}

\paragraph{}

Cell signalling is an important process that regulates various physiological and pathological functions in cells. The study of cell signalling has been the subject of intense research in recent decades, with the development of various mathematical models to understand and analyze the complex interactions between signalling molecules. The use of mathematical models in cell signalling has provided valuable insights into the mechanisms of cell signalling and their alterations in disease conditions such as cancer. This review will discuss the various mathematical modeling approaches used in cell signalling, including ordinary differential equations (ODEs), Boolean networks, and others. We will also review the current state of knowledge about genetic alterations in signalling pathways in cancer, including receptor tyrosine kinases (RTKs) and downstream signalling.

Mathematical models have become essential for studying cell signalling, providing a powerful means for quantifying and analyzing the interactions between signalling molecules. ODEs have been widely used to model cell signalling due to their ability to describe the time-dependent behaviour of signalling networks \cite{glynn2014mathematical}. ODEs are based on the law of mass action and can be used to model the interactions between signalling molecules, including the formation and degradation of signalling complexes \cite{sible2007mathematical}. One of the advantages of using ODEs in cell signalling is that they can provide a more detailed description of the system compared to other modeling approaches, such as Boolean networks \cite{song2021quantitative}.

Boolean networks are another type of mathematical model used to study cell signalling. Boolean networks are based on binary variables and can be used to model the interactions between signalling molecules in a more simplified manner \cite{schwab2020concepts}. Boolean networks are helpful for studying the qualitative behaviour of signalling networks, such as the activation or inhibition of signalling pathways \cite{bock2014boolesim}. However, Boolean networks have limitations compared to ODEs, as they are less capable of describing the quantitative behaviour of signalling networks.

Cancer is a disease characterized by genetic alterations in signalling pathways, leading to the dysregulation of normal cellular processes. RTKs are a class of transmembrane receptors that play a crucial role in cell signalling and are frequently altered in cancer \cite{sever2015signal}. RTKs can activate downstream signalling pathways, leading to the activation of downstream signalling molecules and the regulation of various cellular processes \cite{martin2003cell}. The dysregulation of RTKs in cancer can activate downstream signalling pathways, leading to the development and progression of cancer.

Mathematical models of cell signalling can provide valuable insights into the stability and equilibrium of signalling networks. Linear stability analysis is a commonly used method for analyzing the stability of mathematical models, as it provides information about the long-term behaviour of the system \cite{bardwell2007mathematical}. Eigenvalue analysis is another technique used to analyze the stability of mathematical models, as it provides information about the eigenvalues and eigenvectors of the system \cite{song2021quantitative}. These techniques can be used to study signalling networks' stability and understand the mechanisms underlying their behaviour \cite{tian2017inference}.

The intracellular signalling network involved in apoptosis is another area of interest in cell signalling. The STAT1, STAT3, Bcl-2, and BAX proteins play crucial roles in this network and are the focus of several studies.  used ODEs to model the intracellular signalling network involved in apoptosis and analyzed the impact of the interactions between STAT1, STAT3, Bcl-2, and BAX on the network. Mathematical equilibrium and stability analysis is an essential tool for understanding the behaviour of cell signalling pathways. \cite{mat2019mathematical} used eigenvalue analysis to study the stability of the intracellular signalling network involved in RTK activation. 

Despite the progress made in the field of cell signalling, there are still many limitations in our understanding of these pathways. Further research is needed to address these limitations and improve our understanding of cell signalling pathways. In conclusion, mathematical modeling approaches, including ODEs, Boolean networks, and others, have been widely used in the field of cell signalling to understand the complex interactions between signalling pathways.